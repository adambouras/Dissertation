%!TEX program = xelatex
\documentclass[]{dukedissertation}

%------------------------------------------------------------------------------
% Main packages
%------------------------------------------------------------------------------
\usepackage{amsmath, amssymb, amsfonts, amsthm}
\usepackage{fontspec}
\usepackage{xunicode}
\usepackage{url}
\usepackage[american]{babel}
\usepackage[babel]{csquotes}
\usepackage{color}
\usepackage{bm}
\usepackage{subfig}
\usepackage{graphicx}
\usepackage{mathabx}
\usepackage{multirow}
\usepackage{setspace}
\usepackage{longtable}
\usepackage{booktabs}
\usepackage{pbox}          % For multi-line table cells
\usepackage{pdflscape}     % For landscape PDF pages
\usepackage{unicode-math}  % For custom math fonts
\usepackage[all]{nowidow}  % No widows


% ZOMG!!!1!
% For whatever reason, either bidi or polyglossia redefine \maketitle,
% @makechapterhead, and @makecaption when you load them, which then kills the
% fancy title page stuff that dukedissertation.cls provides. The only way
% around it is this dumb hack, found at
% https://tex.stackexchange.com/a/149295/11851:
%
% - Save the original macros as \kept<macro-name-here>
% - Load polyglossia and let it do its destructive thing
% - Right before \begin{document}, redefine \<macro-name-here> with the 
%     original one that was saved as \kept<macro-name-here>
\let\keptmaketitle\maketitle
\makeatletter
\let\kept@makechapterhead\@makechapterhead
\let\kept@makecaption\@makecaption
\makeatother


%------------------------------------------------------------------------------
% Title page stuff
%------------------------------------------------------------------------------
\author{Andrew Heiss}
\title{Amicable Contempt: The Strategic Balance between Dictators
       and International NGOs}
\supervisor{Judith Kelley}
\department{Sanford School of Public Policy}
\date{2017}

\copyrighttext{All rights reserved except the rights granted by the\\
  \href{https://creativecommons.org/licenses/by-nc-sa/4.0/}
       {Creative Commons Attribution-Noncommercial-ShareAlike 4.0 Licence}
}

\member{Tim Büthe}
\member{Tana Johnson}
\member{Timur Kuran}


%------------------------------------------------------------------------------
% References, hyperlinks, and language stuff
%------------------------------------------------------------------------------
\usepackage[pdfusetitle, plainpages=false, 
            bookmarks, bookmarksnumbered, noibid,
            colorlinks, linkcolor=black, citecolor=black,
            filecolor=black, urlcolor=black]{hyperref}

\hypersetup{pdfinfo={
  Title={Amicable Contempt: The Strategic Balance between Dictators
       and International NGOs},
  Author={Andrew Heiss},
  Creator={Markdown, pandoc, and TeX},
  Subject={Public policy}
  % Keywords={things, go, here}
}}


% polyglossia has to come before biblatex and after hyperref
\usepackage{polyglossia}
\setmainlanguage{english}
\setotherlanguages{arabic, russian}
\usepackage[space]{xeCJK}  % polyglossia doesn't support Chinese but xeCJK does


% Font settings
\defaultfontfeatures{Ligatures=TeX}
\setmainfont[Mapping=tex-text]{Linux Libertine O}
\setsansfont[Mapping=tex-text]{Source Sans Pro}
\setmonofont[Mapping=tex-ansi, Scale=MatchLowercase]{InconsolataGo}
\setmathfont{Libertinus Math}
\newfontfamily\arabicfont[Script = Arabic]{Amiri}
\newfontfamily\cyrillicfont{Linux Libertine O}
\setCJKmainfont{Noto Serif CJK SC}


% Bibliography
\usepackage[authordate, backend=biber,
            autolang=hyphen, bibencoding=inputenc,
            strict, isbn=false, uniquename=false]{biblatex-chicago}

\addbibresource{/Users/andrew/Dropbox/Readings/Papers.bib}


%------------------------------------------------------------------------------
% Other custom commands and settings
%------------------------------------------------------------------------------
% Scale all images to the full text width
%\setkeys{Gin}{width=1\textwidth}


% Number subsubsections
\setcounter{secnumdepth}{3}


% Nicer caption sizes
\captionsetup{font={small, onehalfspacing}, labelfont=bf}
\captionsetup[sub]{font={small, onehalfspacing}, labelfont=bf}


% Command for manually adding hanging indents
\newcommand{\hangstart}%
{\setlength{\parindent}{-15pt}%
\setlength{\leftskip}{15pt}%
% \setlength{\parskip}{4pt}%
\noindent}


% Don't typeset URLs in a monospaced font
\urlstyle{same}
% Add - to url's default breaks
\def\UrlBreaks{\do\.\do\@\do\\\do\/\do\!\do\_\do\|\do\;\do\>\do\]%
    \do\)\do\,\do\?\do\&\do\'\do+\do\=\do\#\do-}


% pandoc makes tightlists sometimes for whatever reason
\providecommand{\tightlist}{%
  \setlength{\itemsep}{0pt}\setlength{\parskip}{0pt}}


% pandoc will not convert text within \begin{} XXX \end{} to Markdown and will
% treat it as regular TeX. Because of this, it's impossible to do stuff like
% this:

% \begin{landscape}
%
% | One | Two   |
% |-----+-------|
% | my  | table |
% | is  | nice  |
%
% \end{landscape}
%
% Since it'll render like: | One | Two | |—–+——-| | my | table | | is | nice |
% 
% BUT, from this http://stackoverflow.com/a/41945462/120898 we can get around
% this by creating new commands for \begin and \end, like this:
\newcommand{\blandscape}{\begin{landscape}}
\newcommand{\elandscape}{\end{landscape}}

% \blandscape
%
% | One | Two   |
% |-----+-------|
% | my  | table |
% | is  | nice  |
%
% \elandscape

% Same thing, but for generic groups
% But can't use \bgroup and \egroup because those are built-in TeX things
\newcommand{\stgroup}{\begingroup}
\newcommand{\fingroup}{\endgroup}


% Restore the commands that polyglossia broke
\let\maketitle\keptmaketitle
\makeatletter
\let\@makechapterhead\kept@makechapterhead
\let\@makecaption\kept@makecaption
\makeatother

% Remove left margin in lists inside longtables
% https://tex.stackexchange.com/a/378190/11851
\usepackage{enumitem}
\usepackage{etoolbox}
\AtBeginEnvironment{longtable} {
    \setlist[itemize]{nosep, wide=0pt, leftmargin=*, 
                      before=\vspace*{-\baselineskip}, 
                      after=\vspace*{-\baselineskip}}}


%------------------------------------------------------------------------------
% Start document
%------------------------------------------------------------------------------
\begin{document}

%------------------------------------------------------------------------------
% TITLE PAGE -- provides UMI abstract title page and copyright note
%------------------------------------------------------------------------------
\maketitle

%------------------------------------------------------------------------------
% ABSTRACT -- start file with '\abstract'.
%------------------------------------------------------------------------------
%\include{{./Abstract/abstract}}

%------------------------------------------------------------------------------
% FRONT MATTER
%------------------------------------------------------------------------------
\tableofcontents
\listoftables
\listoffigures
%\include{{./Abbreviations/listofabbr}} % start file with '\abbreviations'

%------------------------------------------------------------------------------
% ACKNOWLEDGEMENTS -- start file with '\acknowledgements'
%------------------------------------------------------------------------------
%\include{{./Acknowledgements/acknowledgements}}

%------------------------------------------------------------------------------
% MAIN BODY OF PAPER
%------------------------------------------------------------------------------
% This is generated by compiling the individual Markdown files for each chapter
% with ../Makefile (`make duke`)
\include{{./compiled}}

%------------------------------------------------------------------------------
% APPENDICES
%------------------------------------------------------------------------------
\appendix
\include{{./appendix}}

%------------------------------------------------------------------------------
% BIBLIOGRAPHY 
%------------------------------------------------------------------------------
\cleardoublepage
\normalbaselines  % Fixes spacing of bibliography
\printbibliography[heading=bibintoc, title=Bibliography]

%------------------------------------------------------------------------------
% BIOGRAPHY -- start file with '\biography'
%------------------------------------------------------------------------------
%\include{{./Biography/biography}}

%------------------------------------------------------------------------------
% All done \(•◡•)/
%------------------------------------------------------------------------------
\end{document}
